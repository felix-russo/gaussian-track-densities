\documentclass[12pt,a4paper]{scrarticle}

\usepackage{theme/acts}
\usepackage{mathtools}

\title{Gaussian Track Densities}
\author{Felix Russo}

%Define Environments
\newtheorem*{remark}{Remark}
\newtheorem*{definition}{Definition}
\newtheorem{theorem}{Theorem}
\theoremstyle{definition}
\newtheorem{proc}{Procedure}
%
%Other useful stuff
\newcommand\numberthis{\addtocounter{equation}{1}\tag{\theequation}} %to number a single line in gather*
%Shortcuts for Mathematical Expressions
%Use operatorname instead of text to get non-italic variables
%Units
\newcommand{\m}{\text{ m}}
\newcommand{\cm}{\text{ cm}}
\newcommand{\mm}{\text{ mm}}
\newcommand{\mum}{\ \mu \text{m}}
\newcommand{\GHz}{\text{ GHz}}
\newcommand{\dB}{\text{ dB}}
%General
\renewcommand{\exp}[1]{\text{exp}\left(#1\right)}
\newcommand{\e}[1]{\mathrm{e}^{#1}}
\newcommand{\imagunit}{\mathrm{i}}
\newcommand{\real}[1]{\Re\left\{ #1 \right\}}
\newcommand{\imag}[1]{\Im\left\{ #1 \right\}}
\newcommand{\realnumbers}{\mathbb{R}}
\newcommand{\complexnumbers}{\mathbb{C}}
\newcommand{\der}[1]{\partial_{#1}}
\newcommand{\dtheta}{\partial_\theta}
\newcommand{\dt}{\partial_t}
\newcommand{\diff}{\mathcal{D}}
\newcommand{\expdiff}[1]{\mathcal{D}^{#1}}
%Linear Algebra
\newcommand{\scalarproduct}[2]{\left( #1, #2 \right)}
\newcommand{\Tr}[1]{\operatorname{Tr}\left[ #1 \right]}
\newcommand{\bigbracksTr}[1]{\operatorname{Tr}\big[ #1 \big]}
\newcommand{\unity}{\mathds{1}}
\renewcommand{\vec}[1]{\mathbf{#1}}
\newcommand{\greekvec}[1]{\bm{#1}}
\newcommand{\unityvec}[1]{\vec{\hat{e}}_{#1}}
\newcommand{\transpose}{\mathrm{T}}
\newcommand{\mat}[1]{\mathbf{#1}}
%vector calculus
\newcommand{\diver}[1]{\vec{\nabla} \cdot \vec{#1}}
\newcommand{\rot}[1]{\vec{\nabla} \times \vec{#1}}
\newcommand{\grad}{\vec{\nabla}}
%Electro Dynamics
\newcommand{\EField}{\vec{E}}
\newcommand{\DField}{\vec{D}}
\newcommand{\HField}{\vec{H}}
\newcommand{\BField}{\vec{B}}
%Statistics
\newcommand{\CovMat}[1]{\mat{K}_{#1 #1}}
%Tracking
\newcommand{\ip}{\hat{\vec{r}}}
\newcommand{\ipCoord}{\vec{r}}
\newcommand{\ipCoordShifted}{\overline{\vec{r}}}

\begin{document}

\maketitle

\begin{abstract}
    %% Only use basic LaTeX markup here, it gets rendered by MathJax.

This is a whitepaper example. It contains a number of example
patterns, layouts etc.
Simple math like $a + b = c$ or even $\sqrt{s} = 14$ TeV is supported!

Quisque ullamcorper placerat ipsum. Cras nibh. Morbi vel justo vitae lacus
tincidunt ultrices. Lorem ipsum dolor sit amet, consectetuer adipiscing elit. In hac
habitasse platea dictumst. Integer tempus convallis augue. Etiam facilisis. Nunc
elementum fermentum wisi. Aenean placerat. Ut imperdiet, enim sed gravida
sollicitudin, felis odio placerat quam, ac pulvinar elit purus eget enim. Nunc vitae
tortor. Proin tempus nibh sit amet nisl. Vivamus quis tortor vitae risus porta
vehicula.
\end{abstract}

\tableofcontents

 \section{Maximum}
Given the impact parameters $\ip \coloneqq (d_0, z_0, t_0)$ of a track in Perigee parametrization, one can model the probability of the particle passing exactly through a point $\ipCoord = (d, z, t)$ using a multivariate Gaussian distribution \cite{multivariate-gaussian}:
\begin{align*}
    P(d, z, t) \propto \left(\det \CovMat{\ip}\right)^{-\frac{1}{2}}\ \exp{- \frac{1}{2} \ \ipCoordShifted^{\transpose} \ \CovMat{\ip}^{-1} \ \ipCoordShifted},
\end{align*}
%
where $\ipCoordShifted \coloneq \ipCoord - \ip$ and $\CovMat{\ip}^{-1}$ is the impact parameter covariance matrix. In the following, we will calculate the maximum of $P(0, z, t)$, as we only consider the track density along the $z$-axis in the vertex seeding. Since exponential functions are monotonic, we can find the maximum by maximizing the exponent. Due to the minus sign of the latter, the problem reduces to 
\begin{align}
\label{eq:target}
    \ipCoordShifted^{\transpose} \ \CovMat{\ip}^{-1} \ \ipCoordShifted \big|_{d = 0}\rightarrow \text{min}.
\end{align}
%
To minimize a multivariate function, we need to identify its critical points. As we will see in the following, the expression above corresponds to a convex quadratic function. Therefore, it will have exactly one critical point which will correspond to its minimum. To obtain the critical point, we need to solve
\begin{align*}
    \grad_{z, t} \ \ipCoordShifted^{\transpose} \ \CovMat{\ip}^{-1} \ \ipCoordShifted \big|_{d = 0} = 0.
\end{align*}
%
Denoting the entry in the $i$th row and $j$th column of $\CovMat{\ip}^{-1}$ as $W_{ij}$ and writing the impact parameters as $\overline{z} \coloneq z - z_0$, we find:
\begin{align*}
    \ipCoordShifted^{\transpose} \ \CovMat{\ip}^{-1} \ \ipCoordShifted \big|_{d = 0} = W_{11} d_0^2 + W_{22} \overline{z}^2 + W_{33} \overline{t}^2 - 2 W_{12} d_0 \overline{z} - 2 W_{13} d_0 \overline{t} + 2 W_{23}  \overline{z} \overline{t},
\end{align*}
where we used the symmetry of the weight matrix (i.e., the inverse covariance matrix). Noting that $\der{z} = \der{\overline{z}}$, the conditions for the critical point read
\begin{align*}
    W_{22} \overline{z} - W_{12} d_0 + W_{23} \overline{t} &= 0 \\
    W_{33} \overline{t} - W_{13} d_0 + W_{23} \overline{z} &= 0
\end{align*}
Solving these equations yields
\begin{align*}
    \overline{z}^* &= \frac{W_{12} W_{33} - W_{13} W_{23}}{W_{22} W_{33} - W_{23}^2} \ d_0 \\
    \overline{t}^* &= \frac{W_{13} W_{22} - W_{12} W_{23}}{W_{22} W_{33} - W_{23}^2} \ d_0,
\end{align*}
%
where we observe the symmetry of the variables as imposed by \autoref{eq:target}.
\section{Width in z-direction}
We can find the width of the peak in z direction by fixing $\overline{t} = \overline{t}^*$ and treating the exponent as a function of $z$. Then,
\begin{align*}
\label{eq:target}
    &\hphantom{=} -\frac{1}{2} \ipCoordShifted^{\transpose} \ \CovMat{\ip}^{-1} \ \ipCoordShifted \big|_{d = 0,\ \overline{t} = \overline{t}^*} \\
    &= -\frac{1}{2} \left(W_{22} \overline{z}^2  + 2 (W_{23} \overline{t} - W_{12} d_0) \overline{z} + W_{11} d_0^2 + W_{33} {\overline{t}^*}^2 - 2 W_{13} d_0 \overline{t}^* \right) \\
    &= -\frac{1}{2} \Big(W_{22} z^2  + 2 (W_{23} \overline{t}^* - W_{12} d_0 - z_0) z \\ 
    &\hphantom{=} - 2 (W_{23} \overline{t}^* - W_{12} d_0) z_0 + W_{22} z_0^2 + W_{11} d_0^2 + W_{33} {\overline{t}^*}^2 - 2 W_{13} d_0 \overline{t}^* \Big) \\
    &= -\frac{1}{2} W_{22} z^2 - (W_{23} \overline{t}^* - W_{12} d_0 - z_0) z \\ 
    &\hphantom{=} + (W_{23} \overline{t}^* - W_{12} d_0) z_0 -\frac{1}{2} (W_{22} z_0^2 + W_{11} d_0^2 + W_{33} {\overline{t}^*}^2) + W_{13} d_0 \overline{t}^* \\
    &\equiv \alpha  z^2 + \beta z + \gamma,
\end{align*}
where \begin{align*}
    \alpha &\coloneqq -\frac{1}{2} W_{22} \\
    \beta &\coloneqq - (W_{23} \overline{t}^* - W_{12} d_0 - z_0) \\
    \gamma &\coloneqq (W_{23} \overline{t}^* - W_{12} d_0) z_0 -\frac{1}{2} (W_{22} z_0^2 + W_{11} d_0^2 + W_{33} {\overline{t}^*}^2) + W_{13} d_0 \overline{t}^*.
\end{align*}
%
Completing the square furnishes 
\begin{align*}
    -\frac{1}{2} \ipCoordShifted^{\transpose} \ \CovMat{\ip}^{-1} \ \ipCoordShifted \big|_{d = 0,\ \overline{t} = \overline{t}^*} = \alpha \left( z + \frac{\beta}{2 \alpha}\right)^2 + \gamma -\frac{\beta^2}{4 \alpha}.
\end{align*}
%
A comparison to a one dimensional Gaussian yields
\begin{align*}
    \sigma 
    &= \frac{1}{\sqrt{- 2 \alpha}} \\
    &= \frac{1}{\sqrt{W_{22}}},
\end{align*}
and thus \cite{fwhm}
\begin{align*}
    \text{FWHM}
    &= 2 \sqrt{2 \ln{2}} \sigma \\
    &=  2 \sqrt{\frac{2 \ln{2}}{W_{22}}}. 
\end{align*}

\printbibliography{}

\end{document}
